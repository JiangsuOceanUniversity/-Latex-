\def\usewhat{xelatex}
\documentclass[]{theme}
                                                  
% \usepackage{silence}
% % 忽略所有警告(除了错误)
% \WarningFilter{latex}{}{}

\usepackage{lipsum}
\usepackage{makecell}
\usepackage{hong}
\usepackage{tikz}
\usepackage{eso-pic}


\paperType{本科毕业设计} % 你的论文是本科毕业设计 / 本科毕业论文
\ctitle{你的论文标题}    
\EnglishTitle{English Title}  
\caffil{计算机工程学院} % 学院名称
\csubject{计嵌1211}   % 专业名称
\cauthor{你的名字}            % 学生姓名
\cnumber{2000000000}        % 学生学号
\csupervisor{蔡徐坤}        % 导师姓名
\crank{院士}              % 导师职称
\cdate{\the\year~年~\the\month~月~\the\day~日}


% 开始全文
\begin{document}                 

%%%%%%%%%%   封面   %%%%%%%%%%
\makecover
%%%%%%%%%%   原创性声明和使用授权书   %%%%%%%%%%
\makesm
% %%%%%%%%%%   毕业设计中文摘要   %%%%%%%%%%
\ChineseAbstract

\textbf{摘\quad 要:}




\textbf{关键词:} 关键词1;\quad  关键词2; \quad 关键词3; \quad  关键词4

\newpage




% %%%%%%%%%%   毕业设计外文摘要   %%%%%%%%%%
% 1. 外文摘要用Times New Roman字体。
% 2. 标题为四号加粗居中,缩放、间距、位置标准,无首行缩进,无左右缩进,且前空(四号)一行,段前、段后各0.5行间距,行间距为1.25倍多倍行距;
% 3. “Abstract”“Keywords”两词为小四号加粗,顶格,“:”后空一英文字符空格;
% 4. Abstract及Keywords部分的正文为小四号,不加粗,缩放、间距、位置标准,无左右缩进,无悬挂式缩进,段前、段后间距无,行间距为1.25倍多倍行距;
% 5. Abstract部分正文在标题下空一行;
% 6. Keywords部分正文在Abstract部分下空一行,各关键词之间用分号“;”隔开,尾部无标点。
\EnglishAbstract{


\noindent\textbf{Abstract:}  \lipsum[1]

\noindent\textbf{Keywords:} 

}
\newpage



% 去掉单个板材  带研究
%% 多刀具移动和协同运行方法研究
%多板材 
% 物料切割经济模型研究
%%%%%%%%%%   目录   %%%%%%%%%%
\setcounter{page}{1}  
\tableofcontents                                     % 中文目录
\newpage

\makeFBQD
\newpage
%%%%%%%%%  正文  %%%%%%%%%
\section{绪论}
\subsection{xxxx}
\subsubsection{xxxx}



\newpage
%%%%%%%%%%  结论与展望  %%%%%%%%%%
\section*{结论与展望}
% 手动将标题添加到目录中
\addcontentsline{toc}{section}{结论与展望}





\newpage
%%%%%%%%%%  致谢  %%%%%%%%%%
\section*{致谢}
\addcontentsline{toc}{section}{致谢}



\newpage
%%%%%%%%%%  参考文献  %%%%%%%%%%
\begin{thebibliography}{99}

\bibitem {deng} 邓建松,彭冉冉,陈长松. \LaTeX 科技排版指南 [M]. 北京:科学出版社,2001. 书号: 7 - 03 - 009239 - 2/TP.1516



\end{thebibliography}




% \section*{附录}
% \addcontentsline{toc}{section}{附录}


\end{document}                              
